\documentclass[12pt]{article}
\begin{document}

To demonstrate the power of this formalism further, let us compute a specific
example. In terms of $\rho$ we take an atomic measure with
$\rho=p\delta_a+(1-p)\delta_{-a}$, i.e. peaked as $\delta$-functions at $a,-a$.
In terms of $\phi$ we have
\eqn{e6}{\phi(f)=(p\phi_a+(1-p)\phi_{-a})(f)=pf(a)+(1-p)f(-a)}
where $\phi_a(f)=f(a)$ is the linear map on $L^\infty(\R)$ given by evaluation
at $a$. We can also introduce the linear map $D(f)=f'(0)$. It is not a state
but we can still view it as a (densely defined) map on $L^\infty(\R)$ and make
the convolution product etc as above, i.e. we formally view it in the
convolution algebra. Then the form of $\Delta$ in (\ref{e4}) gives
\eqn{e7}{D^n(f)=(D\tens\cdots\tens D)(\Delta^{n-1}f)={\del\over\del
x_1}|_0\cdots {\del\over\del x_n}|_0 f(x_1+\cdots+x_n)=f^{(n)}(0).}


\end{document}